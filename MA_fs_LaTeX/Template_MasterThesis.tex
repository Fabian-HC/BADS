%%%%%%%%%%%%%%%%%%%%%%%%%%%%%%%%%%%%%%%%%
% Masters/Doctoral Thesis 
% LaTeX Template
% Version 2.5 (27/8/17)
%
% This template was downloaded from:
% http://www.LaTeXTemplates.com
%
% This template is based on a template by:
% Steve Gunn (http://users.ecs.soton.ac.uk/srg/softwaretools/document/templates/)
% Sunil Patel (http://www.sunilpatel.co.uk/thesis-template/)
%
% Template license:
% CC BY-NC-SA 3.0 (http://creativecommons.org/licenses/by-nc-sa/3.0/)
%
%%%%%%%%%%%%%%%%%%%%%%%%%%%%%%%%%%%%%%%%%

%----------------------------------------------------------------------------------------	
%	PACKAGES AND OTHER DOCUMENT CONFIGURATIONS
%----------------------------------------------------------------------------------------
\PassOptionsToClass{openany}{MastersDoctoralThesis}

\documentclass[
12pt, % The default document font size, options: 10pt, 11pt, 12pt
%oneside, % Two side (alternating margins) for binding by default, uncomment to switch to one side
english, % ngerman for German
onehalfspacing, % Single line spacing, alternatives: onehalfspacing or doublespacing
oneside,
%draft, % Uncomment to enable draft mode (no pictures, no links, overfull hboxes indicated)
%nolistspacing, % If the document is onehalfspacing or doublespacing, uncomment this to set spacing in lists to single
%liststotoc, % Uncomment to add the list of figures/tables/etc to the table of contents
%toctotoc, % Uncomment to add the main table of contents to the table of contents
%parskip, % Uncomment to add space between paragraphs
%nohyperref, % Uncomment to not load the hyperref package
headsepline, % Uncomment to get a line under the header
openany
%chapterinoneline, % Uncomment to place the chapter title next to the number on one line
%consistentlayout, % Uncomment to change the layout of the declaration, abstract and acknowledgements pages to match the default layout
]{MastersDoctoralThesis} % The class file specifying the document structure

\usepackage[utf8]{inputenc} % Required for inputting international characters
\usepackage{array,booktabs,ragged2e}
\newcolumntype{R}[1]{>{\RaggedLeft\arraybackslash}p{#1}}
\usepackage{color, colortbl}
\usepackage{multirow}
\definecolor{Gray}{gray}{0.9}
\usepackage[flushleft]{threeparttable}
\usepackage{footnote}
\usepackage{amssymb}% http://ctan.org/pkg/amssymb
\usepackage{pifont}% http://ctan.org/pkg/pifont
\usepackage{graphics}
\usepackage{nicefrac}
\usepackage{lscape}
\usepackage[nottoc]{tocbibind}
\usepackage[T1]{fontenc} % Output font encoding for international characters
\usepackage{mathpazo} % Use the Palatino font by default
\usepackage{booktabs}
\usepackage{subfigure}
\usepackage{bigdelim}
\usepackage{multirow}
\usepackage{adjustbox}
\usepackage{threeparttable}
\usepackage{csquotes}
\usepackage[authordate,bibencoding=auto,backend=biber,natbib]{biblatex-chicago}
\addbibresource{thesis_bib.bib}
\DefineBibliographyStrings{english}{%
  bibliography = {References},
  references = {References},
}
\usepackage{hyperref}
\makeatletter
\newenvironment{chapterappendices}{%
  % Restart section numbering from A, B, C...
  \setcounter{section}{0}
  \renewcommand{\thesection}{Appendix \thechapter \Alph{section}}
  \renewcommand{\thesubsection}{\thechapter\Alph{section}.\arabic{subsection}}

  % Allocate more space for section numbering in the ToC
  \addtocontents{toc}{%
    \protect\patchcmd{\protect\l@section}{2.3em}{7em}{}{}
    \setcounter{tocdepth}{0}  % do not show Appendices
  }
}{%
  % Revert to usual space in the ToC when exiting this environment
  \addtocontents{toc}{%
    \protect\patchcmd{\protect\l@section}{7em}{2.3em}{}{}
        \setcounter{tocdepth}{1} 
  }
}
\makeatother

%----------------------------------------------------------------------------------------
%	MARGIN SETTINGS
%----------------------------------------------------------------------------------------

\geometry{
	paper=a4paper, % Change to letterpaper for US letter
	inner=2cm, % Inner margin
	outer=2cm, % Outer margin
	bindingoffset=2cm, % Change if you to change the binding offset at the left margin
	top=2cm, % Top margin
	bottom=2cm, % Bottom margin
	%showframe, % Uncomment to show how the type block is set on the page
}

%----------------------------------------------------------------------------------------
%	THESIS INFORMATION
%----------------------------------------------------------------------------------------

\thesistitle{Does medieval trade still matter? Reassessment for European regions in the 20th century} % Your thesis title, this is used in the title and abstract, print it elsewhere with \ttitle
\supervisor{Nikolaus Wolf} % Your supervisor's name, this is used in the title page, print it elsewhere with \supname
\secondsupervisor{tba} % Your supervisor's name, this is used in the title page, print it elsewhere with \secsupname
\degree{Master of Science} % Your degree name, this is used in the title page and abstract, print it elsewhere with \degreename
\author{Fabian Salger} % Your name, this is used in the title page and abstract, print it elsewhere with \authorname
\studentid{573366} % Your student ID, print it with \id
\addresses{} % Your address, this is not currently used anywhere in the template, print it elsewhere with \addressname
\keywords{} % Keywords for your thesis, this is not currently used anywhere in the template, print it elsewhere with \keywordnames
\university{\href{https://www.hu-berlin.de/}{Humboldt-Universität zu Berlin}} % Your university's name and URL, this is used in the title page and abstract, print it elsewhere with \univname
\department{\href{https://www.wiwi.hu-berlin.de/de/professuren/vwl/wg}{Institut für Wirtschaftsgeschichte}} % Your department's name and URL, this is used in the title page and abstract, print it elsewhere with \deptname
\faculty{{~}} % Your faculty's name and URL, this is used in the title page and abstract, print it elsewhere with \facname

\AtBeginDocument{
\hypersetup{pdftitle=\ttitle} % Set the PDF's title to your title
\hypersetup{pdfauthor=\authorname} % Set the PDF's author to your name
\hypersetup{pdfkeywords=\keywordnames} % Set the PDF's keywords to your keywords
}
 
\makeatletter
\def\hlinewd#1{%
\noalign{\ifnum0=`}\fi\hrule \@height #1 %
\futurelet\reserved@a\@xhline}
\makeatother

\usepackage{alphalph}
\renewcommand*{\thesubfigure}{%
\alphalph{\value{subfigure}})%
}%

\makeatletter
\newcommand{\enableopenany}{%
  \@openrightfalse%
}

\begin{document}
\sloppy
\frontmatter % Use roman page numbering style (i, ii, iii, iv...) for the pre-content pages

\pagestyle{plain} % Default to the plain heading style until the thesis style is called for the body content

%----------------------------------------------------------------------------------------
%	TITLE PAGE
%----------------------------------------------------------------------------------------

\begin{titlepage}
\begin{center}

\vspace*{.01\textheight}

\begin{figure}
\centering
\includegraphics{husiegel}
\end{figure}

{\scshape\LARGE \univname\par}\vspace{1.5cm} % of
\textsc{\Large \degreename}\\[0.5cm] % Thesis type

\HRule \\[0.3cm] % Horizontal line
{\huge \bfseries \ttitle\par}\vspace{0.3cm} % Thesis title
\HRule \\[1cm] % Horizontal line
 
\begin{minipage}[t]{0.4\textwidth}
\begin{flushleft} \large
\emph{Author:}\\
{\authorname}\\ % Author name
{\id}\\
\end{flushleft}
\end{minipage}
\begin{minipage}[t]{0.4\textwidth}
\begin{flushright} \large
\emph{Supervisors:} \\
{\supname}\\
{\secsupname}\\
\end{flushright}
\end{minipage}\\[1cm]
 
\vfill

\large \textit{A thesis submitted to the Chair of Economic History\\
in partial fulfillment of the requirements for the degree \degreename }\\[1cm]% University requirement
 
\vfill

{\large \today}\\[1cm] % Date
 
\vfill
\end{center}
\end{titlepage}

%----------------------------------------------------------------------------------------
%	DECLARATION PAGE
%----------------------------------------------------------------------------------------

\begin{declaration}
\addchaptertocentry{\authorshipname} % Add the declaration to the table of contents
\noindent
I hereby declare that my thesis is the result of my own work and that I have marked all sources, including online sources, which have been cited without changes or in modified form, especially sources of texts, graphics, tables and pictures.
\break\break
\noindent
I assure that I have not submitted this thesis for any other examination yet.
\break\break
\noindent
I am aware that in case of any breach of these rules procedures concerning fraud or attempted fraud will be taken in accordance with the subject-specific examination regulations and/or the Allgemeine Satzung für Studien- und Prüfungsangelegenheiten (ASSP) or the Allgemeine Satzung zur Regelung von Zulassung, Studium und Prüfung der Humboldt-Universität zu Berlin (ZSP-HU).
\break\break
\noindent Signed:\\
\rule[0.5em]{25em}{0.5pt} % This prints a line for the signature
 
\noindent Date:\\
\rule[0.5em]{25em}{0.5pt} % This prints a line to write the date
\end{declaration}

\cleardoublepage

%----------------------------------------------------------------------------------------
%	QUOTATION PAGE
%----------------------------------------------------------------------------------------

%\vspace*{0.2\textheight}

%\noindent {\itshape  Thank you note}\bigbreak

 
%----------------------------------------------------------------------------------------
%	ABSTRACT PAGE
%----------------------------------------------------------------------------------------

\begin{abstract}
\addchaptertocentry{\abstractname} % Add the abstract to the table of contents
In this thesis

\end{abstract}


%----------------------------------------------------------------------------------------
%	LIST OF CONTENTS/FIGURES/TABLES PAGES
%----------------------------------------------------------------------------------------
\setcounter{tocdepth}{1} % Show subsections

\tableofcontents % Prints the main table of contents

\listoffigures % Prints the list of figures

\listoftables % Prints the list of tables

\bigskip

\begin{center}

\end{center}

%\listofappendices


%----------------------------------------------------------------------------------------
%	THESIS CONTENT - CHAPTERS
%----------------------------------------------------------------------------------------

\mainmatter % Begin numeric (1,2,3...) page numbering

\pagestyle{thesis} % Return the page headers back to the "thesis" style
% Include the chapters of the thesis as separate files from the Chapters folder
% Uncomment the lines as you write the chapters

\chapter{Introduction}

The premise for this thesis is the main result of Wahl (2016), which argues that medieval trade activity led to agglomeration and has caused differences in contemporary regional economic development in Europe. Wahl (2016) contributes to the literature on the origins of the concentration of economic development, identifying a causal link from medieval trade activity over city growth to current regional economic development. 

This thesis in turn examines to what extent the main empirical result from Wahl (2016) can be replicated using a series of historical GDP data. Adapting the methodology applied by Wahl (2016), this procedure adds \textit{time} of dimension corresponding to the historical span of the respective data to the analysis. Obeservational units throughout the analyses are mainly restricted by the availability of data on measures of economic performance. For example, Gross Domestic Product per capita ("GDP p.c.") is collected on regional levels according to the "Nomenclature of Units for Territorial Statistic" ("NUTS"). NUTS is a hierarchical system standard organizing the economic territory of Europe. Initially adopted in 2003, the standard is regularly revised in agreement and cooperation with the member states of the European Union ("EU"). Its revisions amend the standard in administrative cycles of at least three years and may include name changes, region mergers and splits as well as border shifts on any level. All changes are regularly published by Eurostat. Where this thesis combines data with boundaries of different revisions, translation according to the official correspondence tables is done as part of data preparation.

The GDP data used in \autoref{chap:WOLF} is taken from Rosés-Wolf database, version 6 (2020). A benefit of this database is that it provides regional GDP data reaching back to 1900 CE. In addition to GDP, this database also includes employment shares across the three main sectors Industry, Agriculture, and Service as well as time-varying Population and time-constant Area of each regional unit. The accompanying caveat is that the data is structured on NUTS2 level of observational detail. Matching the database with variables and control groups from Wahl(2016) which are available on NUTS3 requires aggregation as a conversion mechanism between the NUTS levels. The respective formula depends on the characteristics of each variable and is reported in \autoref{sec:DataAgg}.

Wahl 2016: medieval trade had statistically significant consequences for economic development --> Self-reinforcing nature of the described agglomeration and spatial concentration processes
--> path-dependency of city development

there is a statistically significant and economically relevant positive relationship between medieval trade activities and contemporary regional economic development.

Fixed Effects Model

Definitions "List of Terms"

\newpage

\chapter{Literature Review}

Wahl 2016: medieval trade had statistically significant consequences for economic development --> Self-reinforcing nature of the described agglomeration and spatial concentration processes
--> path-dependency of city development

This places the paper in the subject field of Economic Geography. 

... Kurzer Abriss wie es zu NEG kommt und was andere Autoren schreiben. Endogenous Growth Path dependency, diversion/conversion, determinants for economic prosperity

The predictions from New Economic Geography with first and second nature causes of agglomeration vs. Wahl 

Several studies focus on first-nature causes of agglomeration and find

other papers combine or isolate second-nature causes of agglomeration
second nature (i.e. man made) causes of agglomeration (e.g. knowledge spillovers) linked to characteristics of medieval trade and trade cities

By connecting the economic development with measures on medieval trade

Detailed historical data would allow for additional nuance in the classification of regional medieval trade activity. 


%\citep{smith1776} had some great insights and \citet{krugman1998}, too.

\newpage

\chapter{Direct Replication Benchmark}

This chapter introduces a direct replication of the regional analysis following Wahl (2016). It lays out the empirical specification of the model and reports replication regression results corresponding to Wahl (2016), table 2. \footnote{A working version of the Wahl dataset was kindly made available by the author. Initial data preparation is therefore limited to matching variables used in the publication with the names given in the available dataset. In some cases this is straightforward, while in others several variables have similar short names and are thus not as easily identified.}

The replication regression runs serve as a reference point for the analyses in the following chapters with GDP data from further sources. From a practical perspective, the replication verifies that the model fits indeed resemble the original regression results as closely as possible. In order to achieve equal results, each control group must be composed of the correct set of variables. Further, the dependent variable, the main independent variables, and the empirical specification of the model must be accurately called.

\section{Data Description Wahl}

Wahl (2016) links medieval trade to measures of current economic development. Because the standard measure for economic performance across Europe is GDP, which is reported according to NUTS, the observations in Wahl (2016) and this chapter refer to NUTS3 regions according to NUTS revision 2006. 

Wahl's approach: medieval data on trade cities is available, but contemporary GDP is not available on a city level.
1) identify cities that are considered to have had important commercial or trade activity in medieval times, that is 1500. The regions these cities are situated in today, are considered trade regions or trade centers and coded accordingly.
2) calculate a distance variable that shows the distance between each region's centroid and the nearest trade city
3) compute variable reporting the number of centuries a city can be considered an important trade city

\subsection{Dependent Variables}
The depth of an observational unit is primarily constrained by the data availability of GDP.
main dependent variable not 
ln GDP pc

\subsection{Independent Variables}

The main categorization   independent variab identify cities that are considered to have had important commercial or trade activity in 1500 around the late medieval period. The regions these cities are situated in today, are considered trade regions or trade centers and coded accordingly.
2) calculate a distance variable that shows the distance between each region's centroid and the nearest trade city
3) compute variable reporting the number of centuries a city can be considered an important trade city
overall, Wahl has 119 trade cities in ten European countries, constituting of 839 NUTS3 regions.
If a region contains at least one city that is classified as a medieval trade city, the dummy variable is coded as '1'

The variable Distance to trade center is captured to provide nuance and a direct test of the 'core-periphery' hypothesis.

Because medieval trade affected development through agglomeration which takes place over centuries, 'centuries of trade' is meant to capture by proxy its path dependency. It is coded to take the value as the no of centuries from the century before the city became an important trade city 

control
.
.
.

\section{Empirical Analysis: Replication Regression}
Effect of medieval trade on contemporary regional development.
Expectation: The effect can partially be explained by its influence on agglomeration patterns.

\newpage

\chapter{Aggregated Regional Data (NUTS2)}
\label{chap:WOLF}
combination and introduction to Wolf GDP data
Have: set of variables from the data collected by Wahl and regionally dispersed along NUTS3, and GDP data going back to 1900 on NUTS2 level.
Idea: Combine datasets to long form time-variant version for each year-region pair, including qualitative controls. The controls are taken from Wahl and data which is collected with respect to the reference period 2009 taken as proxies for the reference years.

\section{Data Aggregation}
\label{sec:DataAgg}
Selection of variables
The respective formula depends on the characteristics of each variable and is reported in 
\subsection{WFM}
wide-form merge

\subsection{LFP}
long-form Panel like

\section{Exploratory Data Analysis}

\subsection{Sample Overview}
Maps
\subsection{Sample Overview}
Graphs

\subsection{Composition of further Variables}
var comp
introduction of Growth variable
Data section ends with the ready-to-use dataset MA nuts2 Data


\section{Empirical Baseline Model Approach}
Models and Fits, general approach:
I set up a baseline model first, in which I closely replicate the models run by Wahl.

including reduced fit

\subsection{Growth since 1900 as dependent variable}

Describe run and results for the Growth since 1900 as dependent var

\subsection{Growth over Observation Periods as dependent variable}

Describe run and results for annual average Growth as dependent var

\subsection{Information Loss in Aggregation}
Analyse potential reasons for statistically insignificant results

\section{Alternative Empirical Approaches}
Make use of the benefits of the deeper data and build the hypothesis that a combination of variables may provide more insights 

\subsection{mining and trade center interaction}
Interaction Term DID and the hypothesis that structural/sectoral shifts since roughly 1980 had a less destructive impact on the overall economic performance as measured by the level of GDP in trade regions than in mining regions without prior trade city exposure. 

\subsection{sector analysis over time employment shares}

\section{X with Outlook}

Outlook: discussion on younger urban centers in Europe and the rest of the World. Results importantly hold for Europe only. What about North America? China, where trade has long history as well? Why would results potentially be applicable and why not.

Double check sample choices: The islands Corse, Sicily, and Sardinia are excluded in the first place -- fair practice

Urban centers or medieval trade cities? Are there large urban centers today that are economically strong and that weren't medieval trade cities? Perhaps in other parts of the world. What are reasons for economic strength there? How applicable is the result?

Q: self reinforcing circular causation, caused by backward and forward linkages of agglomeration and core-periphery patterns

Lay hands on the trade city definition -- Trade Center dummy
In order to consistently match the Control Variables in the dataset constructed, for instance changing the trade city dummy without the respective change in distance to trade city and centuries of trade, I did not change the trade city dummy. Its underlying definition is well argued, but not fixed.

\newpage

\chapter{Conclusion}
 
 
%----------------------------------------------------------------------------------------
%	THESIS CONTENT - APPENDICES
%----------------------------------------------------------------------------------------
\appendix % Cue to tell LaTeX that the following "chapters" are Appendices

% Include the appendices of the thesis as separate files from the Appendices folder
% Uncomment the lines as you write the Appendices

% Appendix Template

\chapter{Data Appendix} % Main appendix title

\label{AppendixA} % Change X to a consecutive letter; for referencing this appendix elsewhere, use \ref{AppendixX}

\begin{table}[!htbp] \centering 
  \caption{Descriptive Statistics, c.p. Wahl (2016)} 
  \label{} 
\begin{tabular}{@{\extracolsep{5pt}}lccccc} 
\\[-1.8ex]\hline 
\hline \\[-1.8ex] 
Statistic & \multicolumn{1}{c}{N} & \multicolumn{1}{c}{Mean} & \multicolumn{1}{c}{St. Dev.} & \multicolumn{1}{c}{Min} & \multicolumn{1}{c}{Max} \\ 
\hline \\[-1.8ex] 
lngdp & 839 & 9.995 & 0.510 & 8.319 & 11.324 \\ 
trade\_city\_final & 839 & 0.142 & 0.349 & 0 & 1 \\ 
lndist\_trade\_city1 & 839 & 0.425 & 0.265 & 0.000 & 1.357 \\ 
trade\_year & 839 & 1.088 & 2.823 & 0 & 13 \\ 
latitude & 839 & 49.460 & 3.088 & 38.245 & 55.939 \\ 
longitude & 839 & 10.228 & 5.012 & $-$4.091 & 25.573 \\ 
altitude & 839 & 279.230 & 320.194 & $-$6.200 & 2,472.600 \\ 
lndist\_border & 839 & $-$0.825 & 1.083 & $-$5.532 & 1.160 \\ 
lndist\_coast & 839 & 0.308 & 1.204 & $-$5.565 & 1.882 \\ 
lndist\_river & 839 & $-$0.675 & 1.322 & $-$7.185 & 1.944 \\ 
neighborgdp & 839 & 14.654 & 12.359 & 0.000 & 62.046 \\ 
tradeneighbor & 839 & 0.558 & 0.756 & 0 & 4 \\ 
tradeneighbor2 & 839 & 1.776 & 1.444 & 0 & 7 \\ 
tradeneighbor3 & 839 & 2.785 & 1.902 & 0 & 9 \\ 
tradeneighbor4 & 839 & 8.143 & 4.149 & 0 & 21 \\ 
capital & 839 & 0.011 & 0.103 & 0 & 1 \\ 
mountain\_region & 839 & 0.479 & 1.022 & 0 & 3 \\ 
mining & 839 & 0.228 & 0.420 & 0 & 1 \\ 
lnarea & 839 & 7.032 & 1.297 & 3.575 & 9.400 \\ 
lnagri & 839 & 0.429 & 0.186 & 0.000 & 0.693 \\ 
university\_1500 & 839 & 0.048 & 0.213 & 0 & 1 \\ 
printingpress\_pre1500 & 839 & 0.199 & 0.400 & 0 & 1 \\ 
bishop & 839 & 0.098 & 0.297 & 0 & 1 \\ 
imperial\_city & 839 & 0.069 & 0.254 & 0 & 1 \\ 
hanse\_city & 839 & 0.108 & 0.311 & 0 & 1 \\ 
residence & 839 & 0.067 & 0.250 & 0 & 1 \\ 
imperialroad & 839 & 0.045 & 0.208 & 0 & 1 \\ 
lndist\_roman & 839 & $-$1.470 & 2.161 & $-$8.594 & 2.160 \\ 
education & 832 & 24.211 & 6.319 & 8.400 & 48.600 \\ 
eqi100 & 839 & 72.130 & 17.163 & 10.180 & 97.610 \\ 
inequality2 & 825 & 1.134 & 0.921 & 0.037 & 8.425 \\ 
patents & 803 & 83.094 & 89.654 & 0.286 & 764.717 \\ 
unemployment & 582 & 8.237 & 3.435 & 1.900 & 19.100 \\ 
lnemp\_comp & 825 & 9.867 & 0.924 & 7.086 & 12.331 \\ 
lnfixed\_cap & 803 & 9.141 & 0.818 & 6.802 & 11.494 \\ 
\hline \\[-1.8ex] 
\end{tabular} 
\end{table} 


%----------------------------------------------------------------------------------------
%	BIBLIOGRAPHY
%----------------------------------------------------------------------------------------
 
%\chapter{References}
\backmatter
\printbibliography[heading=bibintoc]


%----------------------------------------------------------------------------------------

\end{document}  

